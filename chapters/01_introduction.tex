\chapter{Introduction}

\subsection*{The tissue-specific paradigm of cancer biology.}

While many think of cancer as the destruction of cellular regulatory mechanisms resulting in an unregulated, anarchical system, advances in cancer genetics have revealed it to be a highly organized and complex disease.
As such, specific alterations to the normal cell are required for transformation, resulting in these changes displaying strong tissue-specificity.
The phenomenon of tissue-specificity of cancer drivers has been extensively reviewed elsewhere \cite{Sieber2005TissueCancers., Schneider2017, Haigis2019}, but the basic principle is that the effect of a disruption (e.g. a single nucleotide substitution, copy number alteration, dysregulated gene expression, chromosomal rearrangement, etc.) is determined by the preexisting cellular and tissue signaling environments.
As a result, the identification of a panacea for cancer is unlikely, and instead advances in therapeutics will rely upon improved precision medicine where each tumor is treated based on its unique properties.

The goal of the research proposed herein is to study the phenomenon of tissue-specificity via two strategies.
In the first, we will study the oncogene \KRAS{} in various tissues, characterizing its genetic interactions in different cancers (Aim 1) and identifying underlying properties of tissues that determine the potency of its oncogenic mutations (Aim 2).
From these analyses, we will have a better understanding of how the interactions of one gene are dependent upon the broader tissue-specific context.
The second strategy we will employ to understand tissue-specificity of oncogenesis is to analyze the effects of a change to the broader cellular context on the mechanisms capable of driving cancer.
This will be accomplished by identifying the impacts of disrupted epigenetic modifications from dysregulated polycomb repressive complex 2 (PRC2) on a cancer's oncogenic mutations, genetic interactions, and genetic dependencies.


\subsection*{\KRAS{} is a tissue-specific oncogene.}

\kras{} is a member of the Ras family of highly homologous, small GTPases, ubiquitously expressed in humans \cite{Barbacid1987, Prior2012}. 
They operate as GTP-regulated signaling hubs, relaying extracellular signals to key intracellular functions such as growth, proliferation, metabolism, and motility. 
When bound to GTP, \kras{} activates its downstream effectors via protein-protein interactions until it is deactivated by hydrolyzing the GTP to GDP with assistance from a GTPase-activating protein (GAP). The release of GDP, facilitated by a guanine nucleotide exchange factor (GEF), allows \kras{} to again bind a GTP molecule, returning to its active state \cite{Barbacid1987, Johnson2017}. Mutations of \KRAS{} that result in a net increase of the cellular concentration of GTP-bound \kras{} are frequently initiating events in cancer \cite{Kanda2012, Zhang2014a, Li2018}, though are specifically enriched in colorectal adenocarcinoma (COAD, mutated in about 50\% of tumors), lung adenocarcinoma (LUAD, 30\%), multiple myeloma (MM, 20\%), and pancreatic adenocarcinoma (PAAD, 90\%) \cite{Prior2020TheCancer}.
In contrast, \KRAS{} is very rarely mutated in other cancer types, resulting in two groups of tissues: \KRAS{}-sensitive and \KRAS{}-insensitive.

In Aim 1, we propose to study the different oncogenic variants (or "alleles") of \KRAS{} in \KRAS{}-sensitive tissues as a method through which to investigate tissue-specific genetic interactions of an oncogene.
First we will study the ability of various mutational processes to cause each allele as a means to explain the allelic diversity across cancer types.
Then we will identify comutation interactions and genetic dependencies between the \KRAS{} alleles and other genes to measure each variants distinct effect on cancer progression.
These results will highlight the necessity to study cancer drivers at the tissue- and allele-level in order to fully describe their oncogenic properties.

Aim 2 proposes to reveal differences in \KRAS{}-sensitive and insensitive tissues that explain the vast disparity in mutational frequencies across cancer types.
Similar to Aim 1, we will begin by estimating the extent to which active mutational processes contribute to the rate of \KRAS{} mutation.
We will then attempt to model the permissivity of a tissue to mutant \KRAS{} on the basal proteomic and phosphoproteomic state.
Finally, the rare instances of \KRAS{}-insensitive cancers with \KRAS{} mutations will be analyzed to determine if and how they became \KRAS{}-sensitive.
These studies will provide insight into the mechanisms that determine the tissue-specificity of \KRAS{}-driven cancer.


\subsection*{Dysregulated PRC2 augments the epigenetic state of a cell.}

% General information.
PRC2 is a protein complex that suppresses gene expression by mono-, bi-, and trimethylating lysine 27 of histone H3 (H3K27).
It is essential for human development, modulating and sustaining the epigenetics of cells as they grow and differentiate.
The complex is composed of four core subunits, EZH1/EZH2, SUZ12, EED, and RBBP4/7 (also known as RBAP46/48), and two subtypes (PRC2.1 and  PRC2.2) have been described as the association of the core subunits with groups of other proteins.
These auxiliary subunits are thought to regulate the enzymatic activity of EZH1/2 and help recruit the complex to different locations on the chromosomes \cite{VanMierlo2019a, Laugesen2019a}.
Exactly how PRC2 is recruited to specific chromosomal locations is still an activate area of research, though there is a known association with unmethylated CpG islands  \cite{Ku2008, Tanay2007HyperconservedSites., Mendenhall2010GC-richCells., Lynch2012AnRecruitment.}.
Additional studies suggest that transcription factors, non-coding RNA, and histone modifications \cite{Laugesen2019a}, including its own product, the methylation of H3K27, influence the localization of PRC2.

% As a system for understanding changes in context (Aim 3)
Previous studies have revealed instances where the behaviour of oncogenes can change in response to epigenetic changes from PRC2 dysregulation \cite{Kim2015SWI/SNF-mutantEZH2., Fillmore2015EZH2Inhibitors., Serresi2016PolycombCancer., Serresi2018Ezh2Vulnerabilities., Chen2018TargetingMedicine.}.
For example, inhibition of EZH2 in NSCLC cell lines with either an \emph{EGFR} or \emph{SMARCA4} mutation were more susceptible to inhibition of topoisomerase II (TopoII).
If both of these genes were WT, however, EZH2 inhibition resulted in upregulated \emph{SMARCA4}, mediating resistance to TopoII inhibition \cite{Fillmore2015EZH2Inhibitors.}.
These studies suggest that the epigenetic changes due to altered PRC2 activity cause substantial alterations to cellular signaling, augmenting the oncogenic potential of some genes.
Therefore, in Aim 3, we propose to study the impact of dysregulated PRC2 as a mechanism for analyzing the consequences of a change to the signaling environment on the tissue-specificity of oncogenes.
We will start by identifying genes that experience different rates of mutation or demonstrate distinct comutation interactions under abnormal PRC2 activity in human tumor samples.
In addition, we will identify genes with differing levels of dependency in cancer cell lines with dysregulated PRC2.
These analyses to characterize changes to the drivers of cancer under an altered epigenetic state will model the impact of the cellular context on oncogene fitness.
