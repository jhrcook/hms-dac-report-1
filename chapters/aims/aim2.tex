\section{Aim 2. Identify the characteristics that determine sensitivity to oncogenic \KRAS{} mutations.}

\subsection*{Rationale}

Though \KRAS{} is estimated to be mutated in 10\%-14\% \cite{Bailey2018, Prior2020TheCancer} of all cancer, only a few cancer types frequently have oncogenic \KRAS{} mutations.
One explanation for this phenomenon is that the actual mutational event is less likely to occur in some tissues compared to others due to differences in mutagenic processes.
However, this is unlikely because neighboring tissues often demonstrate drastically different rates of \KRAS mutations.
One striking example is how 32\% of lung adenocarcinoma, but only 4\% of small cell lung cancer tumors have a \KRAS{} mutation \cite{Bailey2018, Prior2020TheCancer}.
Additionally, experimental evidence has demonstrated that even the forced expression of mutant \kras{} does not induce oncogenesis in non-permissive tissues including XX.
Instead, we hypothesize that the basal state of a tissue, including the existing signaling architecture, determines its sensitivity to \KRAS{} hyperactivation.
Therefore, I aim to identify such properties of tissues that determine their sensitivity to \KRAS{} oncogenesis.

%%%%%%%%%%%%%%%%%%%%%%%%%%%%%%%%%%%%%%%%%%%
% Aim 2.1
%%%%%%%%%%%%%%%%%%%%%%%%%%%%%%%%%%%%%%%%%%%

\subsection*{Aim 2.1. Determine the extent to which mutational processes affects the frequency of \KRAS{} mutations across the cancers.}

\subsubsection*{Approach}

Similar to Aim 1.1, but use count data instead of averaging of tumor samples.

\subsubsection*{Pitfalls and alternative approaches}

Similar problems as with Aim 1.1.

\subsubsection*{Preliminary data}

Nothing.


%%%%%%%%%%%%%%%%%%%%%%%%%%%%%%%%%%%%%%%%%%%
% Aim 2.2
%%%%%%%%%%%%%%%%%%%%%%%%%%%%%%%%%%%%%%%%%%%

\subsection*{Aim 2.2. Model \KRAS{} sensitivity on the basal signaling of tissue.}

\subsubsection*{Approach}

Proteomics and phosphoproteomics from Olesja.

Different types of metrics: raw proteomics, raw phospho, scaled phospho, include both with interactions, include only interactions.

Linear mixed-effects model, random effects = mouse.
Pre-select predictors as those in signaling pathways.
Latent variable models with regularization.

\subsubsection*{Pitfalls and alternative approaches}

Data is difficult to generate, so only a handful of mice and tissues.

Curse of dimensionality.

Alternatives: nonlinear methods (RF, SVM). Problems: more difficult to interpret; potentially more suspetible to of overfitting (more hyperparameters to tune).

\subsubsection*{Preliminary data}

We have data.
Maybe do some quick things with OP data: num. proteins, num phosphosites, clustering, PCA, etc..

%%%%%%%%%%%%%%%%%%%%%%%%%%%%%%%%%%%%%%%%%%%
% Aim 2.3
%%%%%%%%%%%%%%%%%%%%%%%%%%%%%%%%%%%%%%%%%%%

\subsection*{Aim 2.3. Identify recurring events in insensitive tissues with oncogenic \KRAS{} mutations.}

\subsubsection*{Approach}

There are still KRAS mutants. 
Why tho? 
Are they passengers? 
Are they drivers?
What has to happen to allow them to be drivers?

Collect as many cancer samples as possible.

Cluster cancers by the type of disruptions to KRAS: mutation, CNA, RNA expression.

Comutation analysis.
Include CNA.

Associate with chromatin state. 
Large scale restructuring due to epigenetic changes.

\subsubsection*{Pitfalls and alternative approaches}

Rare.
Not all data for every sample.
Thus less power for analysis.

\subsubsection*{Preliminary data}

Bar-plot of number of samples with enough KRAS mutations.
