\chapter{Abstract}
\normalsize

While cancer is an indiscriminate disease, its causes are not. 
Instead, the drivers of cancer are almost all highly tissue-specific, that is, genes that drive cancer usually only do so in cancers from specific tissues of origin.
Understanding the determinants of such strict oncogenicity is critical to treating cancers based on the unique features of an individual case, the goal of \emph{precision medicine}.
As such, we propose to study tissue-specificity via two strategies: 1) characterising the genetic interactions of a tissue-specific oncogene in different cellular contexts, and 2) analyzing the effects of a change to the cellular context due to epigenetic dysregulation.
The first strategy will be pursued by studying \KRAS{}, a gene with a high rate of mutation in just a handful of cancers.
We will characterize how each oncogenic variant of this gene has distinct tissue-specific genetic interactions.
Further, we will identify possible reasons for its strict tissue-specificity by studying its effect in tissues where it not oncogenic.
The second strategy will consist of analyzing the impact of dysregulated epigenetics caused by aberrant PRC2 activity on the selection permissivity of oncogenes.
Together, this research will provide valuable insight into the causes and effects of tissue-specificity of cancer driver genes.
